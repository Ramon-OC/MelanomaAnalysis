\section{Motivación del proyecto}

Las estadísticas sobre el cáncer de piel muestran que la detección temprana puede mitigar en gran medida el riesgo asociado con las formas comunes e inusuales de la enfermedad.  Los exámenes anuales de cuerpo completo son altamente recomendados para detectar el cáncer de piel. Sin embargo, es crucial que los pacientes estén familiarizados con las señales de advertencia y sepan cómo identificar posibles inicios de la enfermedad. Los profesionales de la salud han desarrollado herramientas y métodos útiles con el objetivo de mejorar la efectividad de los autodiagnósticos. Uno de los métodos más ampliamente utilizados es la técnica conocida como ''ABCDE'' del cáncer de piel \autocite{jensen:2015}. Esta herramienta se basa en la observación de cinco características principales de las lesiones cutáneas sospechosas:

\begin{itemize}
    \item \textbf{Asimetría}: una lesión se describe como asimétrica si al rebanarla por la mitad produce dos formas distintas.
    \item \textbf{Borde}: si el borde del crecimiento es irregular o está indefinido, incrementa la probabilidad de que sea cáncer de piel.
    \item \textbf{Color}: si la zona no presenta un color uniforme, requiere una inspección más detallada. Ya sea debido a la presencia de un gradiente en el tono o a que alguna parte del área exhiba un color diferente al resto, es necesario investigar más a fondo.
    \item \textbf{Diámetro}: si un crecimiento mide más de seis milímetros de extremo a extremo,
    \item \textbf{Evolución}: un área que está cambiando constantemente en tamaño, forma o color merece atención médica.
\end{itemize}

El enfoque de evaluación para este proyecto se centrará en tres de los cinco parámetros del autodiagnóstico dermatológico: asimetría, irregularidad y variación de color. La exclusión de los parámetros de diámetro y evolución se debe a la necesidad de disponer de una amplia gama de imágenes para una medición precisa de la evolución y a la importancia de contar con referencias específicas para obtener una aproximación precisa del diámetro

Todas las imagenes manipuladas fueron recuperadas de un conjunto de datos de dominio público y de libre acceso. Las imagenes están divididas en dos categorías: maligno y benigno, ambas pueden servir como parámetros para el análisis y visualización del comportamiento de los algoritmos. La evaluación sobre el estado del tumor y todos los datos pertenecen a The International Skin Imaging Collaboration \autocite{ISIC:2019}. 